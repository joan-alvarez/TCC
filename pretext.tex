% ----------------------------------------------------------
% ELEMENTOS PRÉ-TEXTUAIS
% ----------------------------------------------------------
\pretextual
% ---
% Capa
% ---
\imprimircapa
% ---

% ---
% Folha de rosto
% (o * indica que haverá a ficha bibliográfica)
% ---
\imprimirfolhaderosto*
% ---

%\clearpage
\imprimirfichacatalografica

% ---
% Inserir folha de aprovação
% ---

% Isto é um exemplo de Folha de aprovação, elemento obrigatório da NBR
% 14724/2011 (seção 4.2.1.3). Você pode utilizar este modelo até a aprovação
% do trabalho. Após isso, substitua todo o conteúdo deste arquivo por uma
% imagem da página assinada pela banca com o comando abaixo:
%
% \includepdf{folhadeaprovacao_final.pdf}
%
\begin{folhadeaprovacao}

  \begin{center}
    {\imprimirautor}
    \vspace*{\fill}\vspace*{\fill}
    \begin{center}
      \ABNTEXchapterfont\bfseries\Large\imprimirtitulo
    \end{center}
    \vspace*{\fill}
	    
    %\hspace{.45\textwidth}
    
    	\begin{center}
    		\vspace*{0.5cm}
    		Esta Monografia foi julgada no contexto da disciplina EEL7890 -- Trabalho de Conclusão de Curso (TCC), e aprovado em sua forma final pelo Programa de Engenharia Elétrica da Universidade Federal de Santa Catarina.
    		\vspace*{0.5cm}
	 		\end{center}
	    
    \vspace*{\fill}
   \end{center}
  
  \begin{center}
    %\vspace*{0.5cm}
    {\large\imprimirlocal},
    {\large\imprimirdata}
    %\vspace*{1cm}
	\end{center}
	        
   \assinatura{\textbf{ Prof. Dr. Eng. Renato Lucas Pacheco} \\ Coordenador de Graduação}
   Banca Examinadora:
   \assinatura{\textbf{Prof. Dr. Eng. \imprimirorientador} \\ Orientador} 
   \assinatura{\textbf{Professor} \\ Convidado 1}
   \assinatura{\textbf{Professor} \\ Convidado 2}
   %\assinatura{\textbf{Professor} \\ Convidado 3}
   %\assinatura{\textbf{Professor} \\ Convidado 4}
	      

\end{folhadeaprovacao}
% ---

% ---
% Dedicatória
% ---
\begin{dedicatoria}
	\vspace*{\fill}
	\centering
	\noindent
	\textit{Este trabalho é dedicado à minha família pela compreensão e apoio nos momentos difíceis da jornada da graduação} \vspace*{\fill}
\end{dedicatoria}
% ---

% ---
% Agradecimentos
% ---
\begin{agradecimentos}
\textcolor{red}{Obrigado}
\end{agradecimentos}
% ---

% ---
% Epígrafe
% ---
\begin{epigrafe}
	\vspace*{\fill}
	\noindent
	\hangindent=5cm \\
 	\textit{`` O homem disse que tinha de ir embora -- antes queria me ensinar uma coisa muito importante: -- Você quer conhecer o segredo de ser um menino feliz para o resto da sua vida? \\ -- Quero -- Respondi. \\ O segredo se resume em três palavras, que ele pronunciou com intensidade, mãos nos meus ombros e olhos nos meus olhos: \\ -- Pense nos outros.''}
 	\begin{flushright}
	Fernando Sabino	
	\end{flushright}		
\end{epigrafe}
% ---

% ---
% RESUMOS
% ---

%% resumo em português
%\setlength{\absparsep}{18pt} % ajusta o espaçamento dos parágrafos do resumo
%\begin{resumo}
%Aqui vai o resumo
%
%\textbf{Palavras-chaves}: latex. abntex. editoração de texto.
%\end{resumo}
%
%% resumo em inglês
%\begin{resumo}[Abstract]
% \begin{otherlanguage*}{english}
%   This is the english abstract.
%   \vspace{\onelineskip}
%
%   \noindent 
%   \textbf{Key-words}: latex. abntex. text editoration.
% \end{otherlanguage*}
%\end{resumo}
%
%% resumo em espanhol
%\begin{resumo}[Resumen]
% \begin{otherlanguage*}{spanish}
%   Acá vá el resumen en español.
%   \vspace{\onelineskip}
%
%   \noindent 
%   \textbf{Palabras clave}: latex. abntex. text editoration.
% \end{otherlanguage*}
%\end{resumo}

% ---
% inserir lista de ilustrações
% ---
\pdfbookmark[0]{\listfigurename}{lof}
\listoffigures
\cleardoublepage
% ---

% ---
% inserir lista de tabelas
% ---
\pdfbookmark[0]{\listtablename}{lot}
\listoftables*
\cleardoublepage
% ---

% ---
% inserir lista de abreviaturas e siglas
% ---
\begin{siglas}
  \item[Celesc] Centrais Elétricas de Santa Catarina
  \item[SE] Subestação
  \item[CBU] Camboriú
  \item[CMB] Camboriú Morro do Boi
\end{siglas}
% ---

% ---
% inserir lista de símbolos
% ---
\begin{simbolos}
  \item[$ \Omega $] Letra grega Ômega
  \item[$ \Delta $] Letra grega Delta
\end{simbolos}

% ---
% inserir o sumario
% ---
\pdfbookmark[0]{\contentsname}{toc}
\tableofcontents*
\cleardoublepage
% ---
%fim dos elementros Pré Textuais


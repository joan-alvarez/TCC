\documentclass[a5paper]{ufsc-thesis}

% ---
% PACOTES
% ---

% ---
% Pacotes fundamentais 
% ---
%\usepackage{lmodern}			% Usa a fonte Latin Modern			
\usepackage[T1]{fontenc}		% Selecao de codigos de fonte.
\usepackage[utf8]{inputenc}		% Codificacao do documento (conversão automática dos acentos)
\usepackage{lastpage}			% Usado pela Ficha catalográfica
\usepackage{indentfirst}		% Indenta o primeiro parágrafo de cada seção.
\usepackage{color}				% Controle das cores
\usepackage{graphicx}			% Inclusão de gráficos
\usepackage{microtype} 			% para melhorias de justificação
\usepackage{mathtools} %para usar equations
\usepackage{xfrac} %para usar \sfrac
\everymath{\displaystyle} %força todas as expressões nem estilo de maior fonte
%\usepackage{pslatex} % Coloca as letras em Times New Roman, mas não nos Títulos de seções e Capítulos 
\usepackage{listings}
\usepackage{listingsutf8}


\graphicspath{{figuras/}}
% ---
		
% ---
% Pacotes adicionais, usados apenas no âmbito do Modelo Canônico do abnteX2
% ---
\usepackage{lipsum}				% para geração de dummy text
% ---

% ---
% Pacotes de citações
% ---
\usepackage[brazilian,hyperpageref]{backref}	 % Paginas com as citações na bibl
\usepackage[alf]{abntex2cite}	% Citações padrão ABNT


% ---
% Informações de dados para CAPA e FOLHA DE ROSTO
% ---
\titulo{Não definido ainda}
\autor{Joan Francisco {}Alvarez Burgos}
\local{Florianópolios, SC-Brasil}
\data{\today}
\orientador{Maurício Valencia {}Ferreira da Luz}
%\coorientador{Equipe \abnTeX}
%\instituicao{
%  Universidade Federal de Santa Catarina -- UFSC
%  \par
%  Centro Tecnológico -- CTC
%  \par
%  Departamento de Engenharia Elétrica -- EEL}
\instituicao{Universidade Federal de Santa Catarina}
\centro{Centro Tecnológico -- CTC}
\programa{Programa de Graduação em Engenharia Elétrica}
\assuntos{Subestações,Sistemas de Potência,Instalações Elétricas,Orçamentos}
\tipotrabalho{Trabalho de Conclusão de Curso}
% O preambulo deve conter o tipo do trabalho, o objetivo, 
% o nome da instituição e a área de concentração 
\preambulo{Monografia submetida ao Programa de Graduação em Engenharia Elétrica da Universidade Federal de Santa Catarina como requisito para aprovação na disciplina EEL7890 -- Trabalho de Conclusão de Curso (TCC).}
% ---


% ---
% Configurações de aparência do PDF final

% alterando o aspecto da cor azul
\definecolor{blue}{RGB}{41,5,195}

% informações do PDF
\makeatletter
\hypersetup{
     	%pagebackref=true,
		pdftitle={\@title}, 
		pdfauthor={\@author},
    	pdfsubject={\imprimirpreambulo},
	    pdfcreator={LaTeX with abnTeX2},
		pdfkeywords={abnt}{latex}{abntex}{abntex2}{trabalho acadêmico}, 
		colorlinks=true,       		% false: boxed links; true: colored links
    	linkcolor=blue,          	% color of internal links
    	citecolor=blue,        		% color of links to bibliography
    	filecolor=magenta,      		% color of file links
		urlcolor=blue,
		bookmarksdepth=4
}
\makeatother
% --- 

% --- 
% Espaçamentos entre linhas e parágrafos 
% --- 

% O tamanho do parágrafo é dado por:
%\setlength{\parindent}{1.3cm}

% Controle do espaçamento entre um parágrafo e outro:
%\setlength{\parskip}{0.2cm}  % tente também \onelineskip

% ---
% compila o indice
% ---
\makeindex
% ---

% ----
% Início do documento
% ----
\begin{document}
% Retira espaço extra obsoleto entre as frases.
\frenchspacing 

% ----------------------------------------------------------
% ELEMENTOS PRÉ-TEXTUAIS
% ----------------------------------------------------------
\pretextual

% ---
% Capa
% ---
\imprimircapa
% ---

% ---
% Folha de rosto
% (o * indica que haverá a ficha bibliográfica)
% ---
\imprimirfolhaderosto*
% ---

%\clearpage
\imprimirfichacatalografica

% ---
% Inserir folha de aprovação
% ---

% Isto é um exemplo de Folha de aprovação, elemento obrigatório da NBR
% 14724/2011 (seção 4.2.1.3). Você pode utilizar este modelo até a aprovação
% do trabalho. Após isso, substitua todo o conteúdo deste arquivo por uma
% imagem da página assinada pela banca com o comando abaixo:
%
% \includepdf{folhadeaprovacao_final.pdf}
%
\begin{folhadeaprovacao}

  \begin{center}
    {\imprimirautor}

    \vspace*{\fill}\vspace*{\fill}
    \begin{center}
      \ABNTEXchapterfont\bfseries\Large\imprimirtitulo
    \end{center}
    \vspace*{\fill}
    
    %\hspace{.45\textwidth}
    
    	\begin{center}
    		\vspace*{0.5cm}
    		Esta Monografia foi julgada no contexto da disciplina EEL7890 -- Trabalho de Conclusão de Curso (TCC), e aprovado em sua forma final pelo Programa de Engenharia Elétrica da Universidade Federal de Santa Catarina.
    		\vspace*{0.5cm}
  		\end{center}
    
    \vspace*{\fill}
   \end{center}
  
  \begin{center}
    %\vspace*{0.5cm}
    {\large\imprimirlocal},
    {\large\imprimirdata}
    %\vspace*{1cm}
  \end{center}
        
   \assinatura{\textbf{ Prof. Dr. Eng. Renato Lucas Pacheco} \\ Coordenador de Graduação}
   Banca Examinadora:
   \assinatura{\textbf{Prof. Dr. Eng. \imprimirorientador} \\ Orientador} 
   \assinatura{\textbf{Professor} \\ Convidado 1}
   \assinatura{\textbf{Professor} \\ Convidado 2}
   %\assinatura{\textbf{Professor} \\ Convidado 3}
   %\assinatura{\textbf{Professor} \\ Convidado 4}
      

\end{folhadeaprovacao}
% ---

% ---
% Dedicatória
% ---
\begin{dedicatoria}
   \vspace*{\fill}
   \centering
   \noindent
   \textit{Este trabalho é dedicado à minha família, minha mãe Ana Iris, meu irmão David, meu pai Hans e meus avós Eliana e Arnoldo que foram tão compreensíveis e me deram tanto apoio nos momentos difíceis da jornada da graduação} \vspace*{\fill}
\end{dedicatoria}
% ---

% ---
% Agradecimentos
% ---
\begin{agradecimentos}
\lipsum[1]
\end{agradecimentos}
% ---

% ---
% Epígrafe
% ---
\begin{epigrafe}
    \vspace*{\fill}
		\noindent
		\hangindent=5cm \\
		\textit{`` O maior erro que um homem pode \\
		cometer é sacrificar a sua saúde a \\
		qualquer outra vantagem.''}
		\begin{flushright}
		Arthur Schopenhauer	
		\end{flushright}		
\end{epigrafe}
% ---

% ---
% RESUMOS
% ---

% resumo em português
%\setlength{\absparsep}{18pt} % ajusta o espaçamento dos parágrafos do resumo
%\begin{resumo}
%Aqui vai o resumo

% \textbf{Palavras-chaves}: latex. abntex. editoração de texto.
%\end{resumo}

% resumo em inglês
%\begin{resumo}[Abstract]
% \begin{otherlanguage*}{english}
%   This is the english abstract.

%   \vspace{\onelineskip}
 
%   \noindent 
%   \textbf{Key-words}: latex. abntex. text editoration.
% \end{otherlanguage*}
%\end{resumo}


% ---
% inserir lista de ilustrações
% ---
%\pdfbookmark[0]{\listfigurename}{lof}
\listoffigures
\cleardoublepage
% ---

% ---
% inserir lista de tabelas
% ---
%\pdfbookmark[0]{\listtablename}{lot}
%\listoftables*
%\cleardoublepage
% ---

% ---
% inserir lista de abreviaturas e siglas
% ---
\begin{siglas}
  \item[Celesc] Centrais Elétricas de Santa Catarina
\end{siglas}
% ---

% ---
% inserir lista de símbolos
% ---
\begin{simbolos}
  \item[$ \Omega $] Letra grega Ômega
  \item[$ \Delta $] Letra grega Delta
\end{simbolos}

% ---
% inserir o sumario
% ---
%\pdfbookmark[0]{\contentsname}{toc}
\tableofcontents*
\cleardoublepage
% ---




% ----------------------------------------------------------
% Introdução
% ----------------------------------------------------------
\chapter*[Introdução]{Introdução} %o asterisco exclui a numeração e retira do sumário
\addcontentsline{toc}{chapter}{Introdução} %este volta a adicionar ao sumário a introdução
\lipsum[1-2]

% ---------------------------------------------------
% Capítulo 1]
% ---------------------------------------------------

\chapter{Equipamentos de Subestações}
\label{chap:equipSE}
\lipsum

% ---------------------------------------------------
% Conclusão
% ---------------------------------------------------
\chapter*[Conclusão]{Conclusão}
\addcontentsline{toc}{chapter}{Conclusão}
\lipsum


\bibliography{refs}
\end{document}



%\begin{figure}[htb]
%	\caption{Sinal de ECG}
%	\centering
%	\includegraphics[width=16cm]{matlab1.pdf}
%	\legend{Fonte: Do autor}
%	\label{fig:matlab1}
%\end{figure}

